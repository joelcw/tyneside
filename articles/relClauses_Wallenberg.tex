%%%%%%%%%%%%%%%%%%%%%%% file template.tex %%%%%%%%%%%%%%%%%%%%%%%%%
%
% This is a general template file for the LaTeX package SVJour3
% for Springer journals.          Springer Heidelberg 2010/09/16
%
% Copy it to a new file with a new name and use it as the basis
% for your article. Delete % signs as needed.
%
% This template includes a few options for different layouts and
% content for various journals. Please consult a previous issue of
% your journal as needed.
%
%%%%%%%%%%%%%%%%%%%%%%%%%%%%%%%%%%%%%%%%%%%%%%%%%%%%%%%%%%%%%%%%%%%
%
% First comes an example EPS file -- just ignore it and
% proceed on the \documentclass line
% your LaTeX will extract the file if required
\begin{filecontents*}{example.eps}
%!PS-Adobe-3.0 EPSF-3.0
%%BoundingBox: 19 19 221 221
%%CreationDate: Mon Sep 29 1997
%%Creator: programmed by hand (JK)
%%EndComments
gsave
newpath
  20 20 moveto
  20 220 lineto
  220 220 lineto
  220 20 lineto
closepath
2 setlinewidth
gsave
  .4 setgray fill
grestore
stroke
grestore
\end{filecontents*}
%
\RequirePackage{fix-cm}
%
\documentclass{svjour3}                     % onecolumn (standard format)
%\documentclass[smallcondensed]{svjour3}     % onecolumn (ditto)
%\documentclass[smallextended]{svjour3}       % onecolumn (second format)
%\documentclass[twocolumn]{svjour3}          % twocolumn
%
\smartqed  % flush right qed marks, e.g. at end of proof
%
\usepackage{mhsetup}
\usepackage{amsmath}
\usepackage{mathtools}
\usepackage{natbib}
\usepackage{graphicx}
\usepackage{float}
\usepackage{qtree}
\usepackage[utf8]{inputenc}
\usepackage{gb4e}
\usepackage[T1]{fontenc}
\usepackage{ tipa }
\bibpunct{(}{)}{,}{a}{}{,}
\newcommand{\noteme}[1]{\noindent \textbf{[[JCW:  #1 ]]}}
\renewcommand{\theequation}{\Alph{equation}}

% Insert the name of "your journal" with
\journalname{Journal of }

\begin{document}

\title{Extraposition is Disapppearing
\thanks{digs, geneva, ucl}}
%\subtitle{Do you have a subtitle?\\ If so, write it here}

%\titlerunning{Short form of title}        % if too long for running head

\author{Joel C. Wallenberg}

%\authorrunning{Short form of author list} % if too long for running head

\institute{Joel C. Wallenberg \at
              Newcastle University \\
              Tel.: +44-(0)191-222-7366\\
              \email{joel.wallenberg@ncl.ac.uk}
}

\date{Received: date / Accepted: date}
% The correct dates will be entered by the editor


\maketitle

\begin{abstract}
stuff
\keywords{syntax \and language change \and Indo-European \and evolutionary dynamics \and treebanks}
% \PACS{PACS code1 \and PACS code2 \and more}
% \subclass{MSC code1 \and MSC code2 \and more}
\end{abstract}

\section{Introduction}
\label{intro}
%Kiparsky 1995, Hopper (Hooper?) & Traugott 200?
This squib presents presents important preliminary results demonstrating the existence of a syntactic change which is many times slower than any other that has previously been reported: the syntactic process of relative clause extraposition is being lost in both Germanic and Romance languages.
%Depending on what is driving the change, it may also be a new class of syntactic change. 
As the results are still preliminary and much of their analysis is still unclear, the main goal of this article is to carefully describe a type of syntactic change which presents problems for both synchronic syntactic theory and the theory of language change, but not to present a comprehensive analysis of every aspect of this change at the present time.
Furthermore, this decline in extraposition has been underway for over a thousand years, and while it is nearing completion in modern Portuguese, the change appears to still be ongoing at the present time. That is to say, the construction in (\ref{construct2}) is currently in decline, and likely on its way to extinction:

\begin{exe}
    \ex %English (constructed):
    \begin{xlist}
         \ex \label{construct1} Someone who you like is coming to the party.
         \ex \label{construct2} Someone is coming to the party who you like.
        \end{xlist}
\end{exe}


This clear result has the following consequence: any analysis of relative clauses (and extraposition) must take into account the fact that the extraposed and \textsl{in situ} variants are, in some sense, ``competing grammars'' \citep[in the sense of][inter alia]{kroch1989, kroch1994}; they must be in competition in language use, as one variant is in the process of replacing the other.
The fact that such slow changes exist, and that their effects on synchronic language use can be demonstrated, should cause all researchers in synchronic syntax to question whether all purported cases of ``optional'' movements are in fact changes in progress.
Perhaps ``optional'' has no coherent meaning in grammar, outside of the meaning of change in progress.

The paper is organised as follows.
First, I report quantative data that relative clause extraposition is being lost in the following languages: English, Icelandic, French, and Portuguese, using seven different parsed diachronic corpora (i.e. treebanks).
Next, I discuss the idea that the extraposed and \textsl{in situ} variants are ``competing grammars'', syntactic variants which are mutually exclusive and competing in language use, and suggest some consequences this has for the analysis of relative clauses.
In section \ref{slow}, I discuss the fact that the extraposed variant is specialized for particularly heavy relative clauses, which is the factor that slows the change down to a degree that it is only observable over a period of a thousand years.
Finally, section \ref{pie} builds on \ref{kiparsky1995} and presents the possibility that this change began back in pre-history, at a time when true relative clauses were first innovated in Romance and Germanic, and has been underway ever since that time.



\section{Decline in Relative Clause Extraposition}
\label{relclause}

The data below show a remarkably slow decline in the frequency of relative clause extraposition from subject DPs in four languages: English, Icelandic, French, and Portuguese. It is worth noting that this type of data is entirely unobtainable without diachronic parsed corpora (i.e. treebanks), which have been annotated at a very high level of accuracy (i.e. they have been hand-corrected by expert syntacticians, not only automatically parsed). For this study, the following parsed diachronic corpora were used:

The decline in extraposition is so slow, in fact, that it would also not have been observable if the corpora of English (collectively) and Icelandic were not at sufficient time-depth; any time-window of only a few hundred years would likely be too small to detect such a change at all.

%agnostic as the rightward/leftward-movement question
%Cite Anton about phonological weight being continuous, not discrete !!!


%extraposition in modern Dutch and German, which has specialized along a grammatical, categorical dimension


\section{Extraposition As Competing Grammars}

\section{Why So Slow?}
\label{slow}

\section{Origin of Relative Clauses}
\label{pie}

%continued presence of CP-CARs, compare to Vedic in Kiparsky



%Cite other paper with Joe

\section{Conclusions}

%The information structural criteria are categorical, but probably not the phonological conditioning 

%Cite Anton about phonological weight being continuous, not discrete !!! Also, Laurel on subject-length

%Cite Laurel's thesis about model of use, and subject length

%Cite Japanese Honorific variation -- thank Caroline -- Anna Strycharz

%t/d in style-shifting -- Jen Smith paper and refs









% For one-column wide figures use
%\begin{figure}
% Use the relevant command to insert your figure file.
% For example, with the graphicx package use
%  \includegraphics{example.eps}
% figure caption is below the figure
%\caption{Please write your figure caption here}
%\label{fig:1}       % Give a unique label
%\end{figure}
%
% For two-column wide figures use
%\begin{figure*}
% Use the relevant command to insert your figure file.
% For example, with the graphicx package use
%  \includegraphics[width=0.75\textwidth]{example.eps}
% figure caption is below the figure
%\caption{Please write your figure caption here}
%\label{fig:2}       % Give a unique label
%\end{figure*}
%
% For tables use
%\begin{table}
% table caption is above the table
%\caption{Please write your table caption here}
%\label{tab:1}       % Give a unique label
% For LaTeX tables use
%\begin{tabular}{lll}
%\hline\noalign{\smallskip}
%first & second & third  \\
%\noalign{\smallskip}\hline\noalign{\smallskip}
%number & number & number \\
%number & number & number \\
%\noalign{\smallskip}\hline
%\end{tabular}
%\end{table}


%\begin{acknowledgements}

%\end{acknowledgements}

% BibTeX users please use one of
%\bibliographystyle{spbasic}      % basic style, author-year citations
%\bibliographystyle{spmpsci}      % mathematics and physical sciences
%\bibliographystyle{spphys}       % APS-like style for physics
\bibliographystyle{linquiry2} %use spbasic for journal of comparative gmc linguistics, etc.
\bibliography{joelrefs}   % name your BibTeX data base

\end{document}
% end of file template.tex

 