%\documentclass[11pt,xcolor=gray,handout]{beamer}
\documentclass[hyperref={pdfpagelabels=false}]{beamer}
\let\Tiny=\tiny
\mode<presentation>{
\usetheme{Frankfurt}
%\usecolortheme{lily}
\usefonttheme{serif}
}
\usepackage{default}
\usepackage{verbatim}
%\usepackage{ucs}
\usepackage[utf8]{inputenc}
\usepackage{gb4e}
\usepackage[T1]{fontenc}
\usepackage{ tipa }
\usepackage{qtree}
\usepackage{synttree}
\usepackage{color}
\usepackage{tree-dvips}
\usepackage[absolute,overlay]{textpos}
%\usepackage{covington-beamer}
\usepackage{lmodern}
\usepackage{natbib}
\usepackage{graphicx}

%\usepackage{memoir}
%\usepackage{relsize}
%\newcommand{\subscript}[1]{\raisebox{-0.25em}{\smaller #1}}
%\logo{\includegraphics[height=0.5cm]{hilogo2.png}}
\setbeamertemplate{footline}[frame number] 
%gets rid of navigation symbols
\setbeamertemplate{navigation symbols}{}

\title{Indirect questions: variation, competition and diachrony}
\author{Joel C. Wallenberg and William van der Wurff\\Newcastle University\\
    \texttt{\{joel.wallenberg,w.a.m.van-der-wurff\}@ncl.ac.uk}}
\institute{}
%\date[]{March 18, 2013 \\ University of Oslo}

\begin{document}

\begin{frame}[plain]
\titlepage
\end{frame}

\begin{frame}
\frametitle{Outline}
\tableofcontents
\end{frame}


\section{Evolution and Acquisition}
%Maybe revise presentation
\subsection{The Blocking Effect}
\begin{frame}
\frametitle{The Blocking Effect}
\begin{itemize}
	\item General cognitive pressure against two forms existing for one function (``doublet''). 
	\item This is the ``blocking effect'', as described for morphosyntactic doublets in Kroch (1994).\nocite{kroch1994}
	\item The possible historical outcomes of a doublet are replacement (one form left) or specialisation (forms diverge in function).
	\item Crucially, they do not coexist forever, nor does one variant die randomly.
	\item Part of the blocking effect is a language acquisition strategy, the ``Principle of Contrast'' (E. Clark 1987); \citep[cf. also][and references therein]{clark1990}. \nocite{clark1987}
\end{itemize}
\end{frame}

\subsection{The Principle of Contrast}
\begin{frame}
\frametitle{The Principle of Contrast}
\begin{itemize}
	\item A strategy that children use in acquiring language: assume that two forms have two meanings (or uses).
		\begin{itemize} \item Synonyms should only be acquired as a last resort.\end{itemize}
	\item Demonstrated many times, in experiments like \citet{markmanwachtel1988}.
		\begin{enumerate}
			\item 20 children
			\item 6 pairs of one familiar item (banana, cow, cup, plate, saw, spoon) and one unfamiliar item (cherry pitter, odd shaped wicker container, lemon wedgepress, radish rosette maker, studfinder, tongs).
			\item \textbf{Control}: ``Show me one''
			\item \textbf{Test}: ``Show me the X'' (X = nonsense syllable)
		\end{enumerate}
	\item Control children pick the unfamiliar object at chance levels, but test children choose unfamiliar objects significantly higher than chance.
\end{itemize}
\end{frame}

%Talk more about Principle of Contrast

\begin{frame}
\frametitle{An Evolutionary Process}
\begin{itemize}
	\item A doublet is two variants competing for finite resources, as in e.g. biological evolution.
		\begin{itemize} \item Instead of competing for something like food, they are competing for use (time in the mouths/brains of speakers) 
					\item Selection operates on the number of times a variant is \textbf{heard} (i.e. and processed) by an acquirer. \end{itemize}
	\item Either one variant has an advantage, and so replaces the other \citep[following a logistic function;][]{nowak2006}.
	\item Or neither variant has an advantage, in which case random walk behaviour ensues (until drift kills one variant, in a finite population model).
	\item But in linguistic doublets, random walk cannot persist indefinitely because of the acquisition pressure of the Principle of Contrast.''
\end{itemize}
\end{frame}

\begin{frame}
\frametitle{An Evolutionary Process}
\begin{itemize}
	\item Therefore, we propose that the Blocking Effect is reducible to Darwinian selection; it is just an evolutionary process.
	\item A doublet resolves in replacement when one form has a selectional advantage.
	\item A doublet resolves in specialisation when neither form has a selectional advantage (or a very small one).
	\item Unlike biology, the Principle of Contrast is built into acquisition and prevents random walk.
	\item In biology, a selectional advantage is a higher probability of reproduction.
	\item In language change, a selectional advantage is a higher probability of a child hearing and acquiring a particular structure.
\end{itemize}
\end{frame}


\section{Competing Grammars}
\begin{frame}
\frametitle{Competing Grammars}
\begin{itemize}
	\item This entire interpretation is dependent on some notion of ``competing grammars'' \citep{kroch1989, kroch1994}
	\item \textbf{Competing Grammars, general form}: 2 variants are available to a speaker with overlapping functions (e.g. the same meaning), and they can't both be used at the same time.
		\begin{itemize}
			\item E.g. they two versions of the same syntactic head, and so occupy the same place in the phrase structure. 
			\item E.g. realisations of the same phoneme.
		\end{itemize}
	\item A concept like this is necessary for any description of a linguistic change in a categorical dimension, like a parameter, where one structure replaces another structure over time. 
		\begin{itemize} 
			\item In any such case, a speaker in the middle of the change (code-)switches between categorical variants.
		\end{itemize} 
\end{itemize}
\end{frame}

\begin{frame}
\frametitle{Polar Qs and Competing Grammars}
\begin{itemize}
	\item The competing grammars situation (like all morphosyntactic doublets) is constrained to two possible outcomes \citep{kroch1994}, as the \textsl{whether/if} study shows:
		\begin{itemize}
			\item Replacement of one variant by the other.
			\item Specialisation of each variant for a specific context over time.
		\end{itemize}
	\item \textbf{Proposal:} every case of syntactic variation or optionality can be reduced to competing grammars, in one of these two versions.
	\item That is good, because competing syntactic heads is the only possible locus of variation/optionality in a BPS syntax.
\end{itemize}
\end{frame}

\subsection{Syntactic Optionality as Competing Grammars}
\begin{frame}
\frametitle{Example: English ``Topicalization''}
\begin{itemize}
	\item \citet{prince1985,prince1998, prince1999}: felicitous in two English discourse contexts, both of which require a certain type of contrast to appear on the fronted XP.
	
	\begin{exe}
\ex \label{princetop1} She's going to use three groups of mice.
One, she'll feed them mouse chow, just the regular stuff they make for
mice.
Another she'll feed them veggies.
And the third she'll feed junk food.\\

\ex \label{princetop2} She was here two years.
[checking transcript] Five semesters she was here.\\
\citep[][8,9]{prince1999} 

\end{exe}

	\item However, it is \textbf{never} obligatory.
\end{itemize}
\end{frame}

\begin{frame}
\frametitle{Example: English Topicalization}
\begin{itemize}
	\item As long as the accent pattern is kept constant, both orders are felicitous:
	
	\begin{exe}
\ex \label{princetop1} She's going to use three groups of mice.
One, she'll feed them mouse chow, just the regular stuff they make for
mice.
Another she'll feed them veggies.
And \textbf{the third} she'll feed \textbf{junk food}.\\

\ex \label{untop1} She's going to use three groups of mice.
One, she'll feed them mouse chow, just the regular stuff they make for
mice.
Another she'll feed them veggies.
And she'll feed \textbf{the third} \textbf{junk food}.\\

\end{exe}

\end{itemize}
\end{frame}

\begin{frame}
\frametitle{Example: English ``Topicalization''}
\begin{itemize}
	\item As long as the accent pattern is kept constant, both orders are felicitous:
	\begin{exe}

\ex \label{princetop2} She was here two years.
[checking transcript] \textbf{Five semesters} she was here.\\

\ex \label{untop2} She was here two years.
[checking transcript] she was here \textbf{five semesters}.\\

\end{exe}

\end{itemize}
\end{frame}

\begin{frame}
\frametitle{Topicalization in Minimalism}
\begin{itemize}
	\item Under current assumptions, Move is triggered by the feature content of some head.
	\item Given ``Merge...preempts Move'' \citep{chomsky2000}, a feature cannot encode optional movement.
	\item Therefore, optional movement must involve a choice between two variants of a functional head (for the numeration), out of an inventory of possible heads:
\end{itemize}

\Tree [.CP XP_i [.C' {C\\$[F]$} \qroof{...t_i...}.TP ] ] \Tree [.CP C \qroof{...XP...}.TP ]

\begin{itemize}
	\item This is the core case of morphosyntactic doublet (i.e. competing heads) described in \citet{kroch1994}.
\end{itemize}

\end{frame}

\subsection{A Minimalist Hypothesis for Variation/Optionality}
\begin{frame}
\frametitle{A Minimalist Hypothesis}
\begin{itemize}
	\item Given that: 
		\begin{itemize} 
		\item these mechanics are necessary to encode syntactic optionality in a Minimalist system,
		\item the same mechanics are necessary to describe a change in progress
		\end{itemize}
	\item Then, the system is simplest if no more machinery is added to deal with optionality/variation.
\end{itemize}

\end{frame}

\begin{frame}
\frametitle{A Minimalist Hypothesis}
\begin{itemize}
	\item \textbf{Prediction:} every case of syntactic optionality or variation is one of the following:
		\begin{enumerate}
			\item A replacement change in progress (outright competition going to completion).
			\item A specialisation change in progress (specialisation for different functions going to completion).
			\item \textbf{The only real case of diachronically stable variation/optionality:} variants have partially specialised along a continuous (or ordinal) dimension, e.g. style, prosodic weight. 
		\end{enumerate}
	\item If categorical variants specialise along a categorical dimension, then complete specialisation should eventually result.
	\item If categorical variants specialise along a continuous or ordinal dimension, then complete specialisation can \textbf{never} result (but replacement can still be arrested).
\end{itemize}

\end{frame}

\begin{frame}
\frametitle{Example: English Topicalization}

\begin{itemize}
	\item Is the frequency stable over time? Probably, at least since Late Middle English \citep{speyer2010}.
	\item Is it specialised for different speech styles (registers)? Not that we know of.
	\item Is it sensitive to prosody? Definitely \citep{speyer2010}.
\end{itemize}

	\begin{exe}

	\ex  The first she'll feed mouse chow, the second she'll feed veggies, and \textbf{the third} she'll feed  \textbf{junk food}.\\

	\ex  ? The first Anders will feed, the second Joel will feed, and the third Wim will feed. \\

	\ex  ?? Joel Anders will pay, Jill Wim will pay, and Ann Maggie will pay. \\

	\end{exe}
\end{frame}

\section{Embedded Polar Questions}

\begin{frame}
\frametitle{Case Study: Theory of Language Change}
\begin{itemize}
	\item What is the \textsl{if/whether} variation in English questions? \\ \citep{baileywallenbergwurff2012}
	\begin{exe}
		\ex John wondered whether Mary was coming to the party.
		\ex John wondered if Mary was coming to the party.
	\end{exe}
	\item Is it pure optionality? Constrained optionality? Some kind of change in progress?
	\item Why don't closely related languages, e.g. Icelandic, show the same variation?
	\item What kind of selective pressures might operate (or not) in language acquisition?
	\item What does this case tell us about syntactic change and variation in general?
	
\end{itemize}
\end{frame}





\subsection{Quantitative Study}

\begin{frame}
\frametitle{A Quantitative Study}
\begin{itemize}
	\item A quantitative study of embedded \textsl{yes/no}-questions in English and Icelandic, comparing the use of \textsl{whether} vs. \textsl{if}, and \textsl{hvort} vs \textsl{ef}.
	\item \textbf{Result 1:} A strong predictor of \textsl{whether} vs. \textsl{if} in both languages is the presence/absence of a disjunction (i.e. \textsl{or, eða}) in the clause, with \textsl{whether} being favoured in the disjunction case more than in the simple case in both languages, across their whole histories.
	\item This indicates a remarkably long-lasting ``persistence'' effect of the original reanalysis environment (inspired by \textsl{have} vs, \textsl{have got} study by Shawn Noble, reported in \citealt{kroch1989}, cf. also \citealt{labov1989}).
	\item The effect supports our hypothesis about the reanalysis environment.
\end{itemize}
\end{frame}


\begin{frame}
\frametitle{A Quantitative Study}
\begin{itemize}
	\item \textbf{Result 2:} The \textsl{whether} structure completely replaces the \textsl{if} structure in the Icelandic case, but not in the English case.
	\item If the two possible outcomes of a morphosyntactic doublet are replacement or specialisation \citep{kroch1994}, Icelandic shows the former and English shows the latter.
	\item We propose that replacement must occur when there is some selectional advantage to one of the variants (in Darwinian terms, where reproduction = learning).
	\item Specialisation must occur when there is no selectional advantage to one of the variants.
	\item \textbf{Experimental Infrastructure:} accurate parsed diachronic corpora: \small{ YCOE \citep{ycoe}, PPCME2 \citep{ppcme2}, PPCEME \citep{ppceme}, PPCMBE \citep{ppcmbe}, and IcePaHC \citep{icepahc09}.}
	
\end{itemize}
\end{frame}

\begin{frame}
\frametitle{English Examples}
\begin{itemize}
\item[ ]\textbf{Disjunction:}
\begin{exe}
	\ex I wonder \{whether,if\} John or Bill is bringing coffee.
	\ex I wonder \{whether,if\} John is bringing tea or coffee.
	\ex I wonder \{whether,if\} John is bringing tea or not.
\end{exe}
\item[ ]\textbf{Simple:}
\begin{exe}
	\ex I wonder \{whether,if\} Bill is bringing coffee.
\end{exe}
\end{itemize}
\end{frame}

\begin{frame}
\frametitle{Icelandic Examples}
\begin{itemize}
\item[ ]\textbf{Disjunction:}
\begin{exe}
	\ex \gll eftir því \textbf{hvort} maður vill heitt eða kalt\\
	according it-DAT whether man wants hot or cold\\
	\quad ``According to whether one wants hot or cold'' \\(\textsl{Sagan Öll}, date: 1985, from IcePaHC, \citealt{icepahc09})
\end{exe}
\item[ ]\textbf{Simple, (older) Icelandic:}
\begin{exe}
	\ex \gll vér vitum eigi, \textbf{hvort} vér tökum öndina\\
	We know not whether we take soul-the\\
	\ex \gll og spurðu, \textbf{ef} hann væri Kristur\\
	and asked if he were Christ\\
	(\textsl{Icelandic Homilies}, date: 1150, from IcePaHC)
\end{exe}
\end{itemize}
\end{frame}

\begin{frame} 
 \frametitle{English \textsl{whether} vs. \textsl{if} Questions, N = 1929 clauses}

\begin{center}
 
 \begin{textblock*}{125mm}(0mm,14mm)
\includegraphics[width=109mm,height=82mm,clip=true,trim=0mm 0mm 0mm 0mm]{whetherifEngSimple.pdf}
\end{textblock*}
%\includegraphics[scale = 0.5]{whetherifEng.pdf}
\end{center}
\end{frame}

\begin{frame} 
 \frametitle{Icelandic \textsl{hvort} vs. \textsl{ef} Questions, N = 397 clauses}
\begin{center}
 
 \begin{textblock*}{125mm}(0mm,14mm)
\includegraphics[width=109mm,height=80mm,clip=true,trim=0mm 0mm 0mm 0mm]{whetherifIceSimple.pdf}
\end{textblock*}

\end{center}
\end{frame}

\begin{frame} 
 \frametitle{English \textsl{whether} vs. \textsl{if}, N = 1929 clauses}

\begin{center}
 
 \begin{textblock*}{125mm}(0mm,14mm)
\includegraphics[width=109mm,height=82mm,clip=true,trim=0mm 0mm 0mm 0mm]{whetherifEng.pdf}
\end{textblock*}
%\includegraphics[scale = 0.5]{whetherifEng.pdf}
\end{center}
\end{frame}

\begin{frame} 
 \frametitle{Effect of disjunction and time period on \textsl{whether} use in English}
 \begin{center}
                    
\begin{tabular}{llllll}
\hline
  & Df & Deviance & Resid. Df & Resid. Dev &  Pr(>Chi) \\
\hline
NULL &   &    &            1928  &   1928.3 &  \\
Disj   &    1 & 152.667  &    1927  &   1775.7 & < 2e-16\\
Time     &    1  &  1.480  &    1926  &   1774.2  & 0.224\\
Disj:Time  & 1  &  5.401   &   1925   &  1768.8 &    \textbf{0.0201}\\
\hline
\end{tabular}
\end{center}
\begin{itemize}
\item A model without an interaction between Disjunction and Time fits significantly worse.
\item Note that there is no clear effect of Time on \textsl{whether} use in general; the interesting effect is an interaction between Time, Disjunction, and  \textsl{whether} use.
\item In other words, \textsl{whether} is not in decline, being replaced by \textsl{if}, but rather they are diverging from each other in use, specializing for the two contexts.
\end{itemize}
\end{frame}

\begin{frame} 
 \frametitle{English, Logistic Model, N = 1929}
 \begin{center}
 \begin{textblock*}{125mm}(0mm,14mm)
\includegraphics[width=109mm,height=80mm,clip=true,trim=0mm 0mm 0mm 0mm]{whetherifEngmodel.pdf}
\end{textblock*}
 \end{center}
\end{frame}

\begin{frame} 
 \frametitle{Icelandic \textsl{hvort} vs. \textsl{ef}, N = 397 clauses}
\begin{center}
 
 \begin{textblock*}{125mm}(0mm,14mm)
\includegraphics[width=109mm,height=80mm,clip=true,trim=0mm 0mm 0mm 0mm]{whetherifIce.pdf}
\end{textblock*}

\end{center}
\end{frame}

\begin{frame} 
 \frametitle{Icelandic, Logistic Model, N = 397}
 \begin{center}
 \begin{textblock*}{125mm}(0mm,14mm)
\includegraphics[width=109mm,height=80mm,clip=true,trim=0mm 0mm 0mm 0mm]{whetherifIcemodel.pdf}
\end{textblock*}
 \end{center}
\end{frame}


\begin{frame}
\frametitle{Whether/if Questions}

\begin{exe}
		\ex John wondered whether Mary was coming to the party.
		\ex John wondered if Mary was coming to the party.
\end{exe}
\begin{itemize}
	\item Is the frequency stable over time? Not in earlier English, but possibly in very recent history.	
	\item Is it specialised for different speech styles (registers)? Yes, according to \citet{biberetal1999}.
	\item Will they ever completely specialise for disjunction/simple? It depends on the strength of the style effect.
\end{itemize}
\end{frame}

\section{Conclusions}
\begin{frame}
\frametitle{Conclusions}
\begin{itemize}
	\item We have shown that plausible reanalysis of \textsl{whether} in disjunctive contexts in Northwest Germanic led to its use in embedded \textsl{yes/no}-questions.
	\item The effect of disjunctive contexts remains a crucial factor conditioning the choice between \textsl{if} and \textsl{whether} over the whole histories of English and Icelandic (in which it is categorical).
	\item \textsl{whether} replaces \textsl{if} in Icelandic, but in English the two items specialise for different functions.
	\item We suggested a reason for the difference between the two languages based on the continuing presence of the original reanalysis environment in Icelandic only (though this hypothesis would require more careful quantitative work to test).
\end{itemize}
\end{frame}

\begin{frame}
\frametitle{Conclusions and Future Research}
\begin{itemize}
	\item The data from the two languages illustrate the two possible outcomes for grammars that come into competition for use, according to the ``Blocking Effect''.
	\item On the basis of this case study, we have suggested that the mechanism of competition for use could underlie all optional or variable syntactic phenomena, given:
		\begin{itemize}
			\item The Principle of Contrast as an acquisition strategy.
			\item The mathematical nature of the dimension along which variants contrast (specialise).
		\end{itemize}
	\item Could we write a realistic, unified algorithm for acquisition, which predicts these effects? 
\end{itemize}

\end{frame}




\begin{frame}[allowframebreaks]
\frametitle{References}
\newcommand*{\newblock}{natbib}
\bibliographystyle{linquiry2}
\bibliography{joelrefs}
\end{frame}




\end{document}